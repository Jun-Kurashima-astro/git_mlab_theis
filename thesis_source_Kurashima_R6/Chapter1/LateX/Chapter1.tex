\chapter{Introduction}

\section{研究背景}



20世紀後半、太陽大気コロナの極端紫外線およびX線分光観測、さらには太陽光エネルギー粒子や太陽風の直接観測により、太陽コロナの元素組成比が下層の光球と異なることが明らかになった。この違いは、元素の第一イオン化ポテンシャル(First Ionization Potential, FIP)に依存することによって説明されている(e.g.,Meyer 1985\cite{Meyer};Schmelz et al.2012\cite{Schmelz};Reames 2018\cite{Reames})。コロナでは、FIPが低い元素($\gtrsim10$)は、光球の元素組成比と比較して2倍から4倍に増加している(e.g.,Feldman 1992\cite{Feldman}; Dennis et al. 2015\cite{Dennis})。これは広く``FIP効果''として知られているが、そのメカニズムは、太陽物理学におけるコロナ加熱問題と並ぶ、長年の未解決問題の一つである。
これらの問題を理解するために、太陽表面からコロナにかけての温度と元素組成比の変化を明らかにすることが重要である。

\section{本研究の目的}
本研究の目的は、太陽フレア発生時の太陽コロナの元素組成比を明らかにすることである。フレア発生時におけるコロナの元素組成比の変動は、FIP効果やコロナ加熱問題の物理的過程やエネルギー輸送のメカニズムを理解する上で重要な手がかりとなる。

Katsuda et al. 2020\cite{Katsuda} は、X線天文衛星 Suzaku を用いた地球大気反射X線の観測により、4つのフレアにおける元素組成比を算出した。フレアごとに元素組成比のばらつきが存在し、このばらつきを検証するには、より多くのフレア観測データが必要となる。2024年は太陽活動の極大期にあたり、2023年10月に観測を開始したX線天文衛星 XRISM によって、多数の大規模フレアが観測されている。本研究では、XRISM /SXI で観測した9つのフレアを対象に元素組成比を算出し、それぞれのフレアにおける元素組成のばらつきを検証する。

また、フレア中の Si の元素組成比は一定であるが、崩壊期に2倍に増加すると報告されている(Katuda et al.2020\cite{Katsuda})。本研究では、フレアを時間分割して元素組成比を算出し、フレア中の Si, S, Ar, Ca, Fe の 元素組成比の変動を検証する。
	
\section{本論文の流れ}
本論文の流れについて説明する。第2章では、太陽大気の基本構造と太陽活動について説明する。第3章では、観測に使用したX線天文衛星XRISM と検出器 SXI について、また 静止気象衛星 GOES と 検出器 XRS について紹介する。第4章では、本研究で解析に用いた観測データについて述べる。第6章では、スペクトル解析の手法とその結果を示す。最後に第7章で研究結果をまとめる。