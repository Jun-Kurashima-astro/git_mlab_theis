\chapter{まとめ}
本研究では、X線天文衛星 XRISM に搭載した X線撮像検出器 SXI を用いて、\mbox{2023年10月19日}から\mbox{2024年12月31日}の期間における、太陽フレアの地球大気反射X線の観測を行った。以下に結果をまとめる。

\begin{itemize}
	\item 9つのフレア観測データを足し合わせ、積分スペクトルを作成した。積分したフレアスペクトルから、Si, S, Ar, Ca, Ti, Cr, Fe, Ni の輝線が得られ、Katsuda et al.2020\cite{Katsuda} と比較して、新たに Ti, Cr, Ni の輝線を確認した。
	
	\item フレア毎に絶対元素組成比を測定した。
	\begin{itemize}
		\item 絶対元素組成比を算出したすべてのフレアが i-FIP を示す結果となった。これは、Katsuda et al.2020\cite{Katsuda}と同様の結果である。
		\item Katsuda et al.2020\cite{Katsuda}と比較して、Si, S, Ar, Ca の絶対元素組成比が高い値となった。この原因として、元素組成比の測定に使用したフレアクラスの違いが大きく影響していると考えられる。Katsuda et al.2020{Katsuda} で使われた4つのフレアは、X17.0, X5.4, X6.2, X9.0 であり、本研究で使用した X1 クラスや、X3 クラスのフレアと比べて、高いクラスのフレアが使用されている。
		\item 本研究では新たに、Fe の絶対元素組成比を算出した。結果として、Fe の絶対元素組成比は、ほとんどのフレアで 0.76-1.37 であった。これは FIP の近い Si と同様の結果であり、恒星コロナで観測される FIP 効果と i-FIP 効果の両方を説明する Laming model (Laming et al, 2021\cite{Laming_2021})と整合している。
	\end{itemize}
	\item フレアピーク付近における絶対元素組成比の時間変動を検証した。
	\begin{itemize}
		\item Si , S, Ca の絶対元素組成比は、フレアの立ち上がりから、フレアピークにかけて増大し、フレアピーク以降は減少している傾向が見られた。特に Ca はフレアピーク付近で顕著に変動することが明らかとなった。
	\end{itemize}
\end{itemize}

