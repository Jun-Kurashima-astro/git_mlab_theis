\chapter{太陽大気の基本構造と太陽活動}
本章では、太陽大気の基本構造と太陽活動について述べる。本章は、谷口(2013)\cite{shin_tenmongaku}、三谷(2005)\cite{Mitani} を参考にした。

\section{太陽大気の基本構造}

図~\ref{fig:solar_tmp}に太陽大気の温度と密度の分布を示す。太陽大気は、内側から順に、光球$\cdot$彩層$\cdot$遷移層$\cdot$コロナと呼ばれる層で構成されている。それぞれの層で温度や密度が大きく異なり、その変化は観測や理論研究によって明らかになっている。
光球の温度は約6,000Kで、そこから高度が上がるにつれて一度温度が下がる。しかし、彩層に達すると温度は再び上昇し、約1万Kに達する。その先の遷移層を経て、コロナでは約200万Kという極めて高温な状態となる。
密度の変化も激しく、光球からコロナにかけて約7桁も減少する。こうした物理的な変化により、異なる高度から異なる波長の光が放射される。そのため、観測する波長によって太陽の姿は大きく異なり、異なる波長で撮影すると太陽の様々な姿を見ることができる。

\begin{figure}[H]
	\centering
	\includegraphics[width=0.7\linewidth]{Chapter2/Figures/ch2_solar_tmp.jpg} 
	\caption{太陽大気の温度・水素密度分布図。0 は太陽光球面を示す。Lang(1995)\cite{Lang_1995}}
	\label{fig:solar_tmp}
\end{figure}

図~\ref{fig:solar_image}に、さまざまな波長で観測した太陽の姿を示す。可視光では黒点が暗く見えるが、紫外線やX線では逆に明るくなる部分もある。これは、太陽の大気が波長によって異なる高度の情報を反映しているためである。太陽内部では核融合反応によってエネルギーが生み出されており、その熱源は中心部に存在する。一般的に、熱源から離れるほど温度は下がると考えられるが、太陽では例外的な現象が起きている。光球と彩層の境界付近に温度が最も低くなる領域があり、そこを超えると逆に温度が上昇する。彩層は約1万K、さらに高度が上がると遷移層を経てコロナでは200万Kにも達する。このような温度の逆転が起こる理由は完全には解明されておらず、``コロナ加熱問題''および``彩層加熱問題''として知られている。

高温の大気が太陽風を生み出し、太陽系全体に影響を及ぼしている。例えば、地球周辺では太陽風と地球磁場がぶつかることで磁気圏が形成され、オーロラや磁気嵐の原因となる。さらに、太陽はG型主系列星に分類されるため、太陽の加熱メカニズムを理解することは、他の恒星の大気構造を知る手がかりにもなる。実際に、コロナを持つ恒星も多数発見されており、太陽と同様の加熱機構が働いている可能性が示唆されている。


\begin{figure}[H]
	\centering
	\includegraphics[width=0.7\linewidth]{Chapter2/Figures/ch2_solar_image.jpg} 
	\caption{さまざまな波長でみた太陽の様子。さ太陽面上を黒点群がトランジットした際のライトカーブ(実線)。背景は、黒点が太陽面中心付近に到達した時刻の画像。黒点トランジットに伴って、可視光ではライトカーブが減光するほか、彩層やコロナに感度を持つ紫外線・X線では増光が生じる。唯一、遷移層に対応する紫外線では、ライトカーブが減光を示す(ISAS/JAXA\cite{solar_image})。}
	\label{fig:solar_image}
\end{figure}



\subsection{光球}

太陽の表面に見える明るい層は``光球(photosphere)''と呼ばれ、その温度は約6,000Kである。光球の厚さは約500kmほどしかなく、これは太陽の半径(約70万km)の1,000分の1程度にすぎない。この薄い層から放射される光が、我々が地球で目にする太陽の光のほとんどを占める。図~\ref{fig:solar_image}の可視光連続光での観測に示すように、光球を可視光で観察すると、中心部が最も明るく、縁に向かうにつれて徐々に暗くなる現象が見られる。これは``周縁減光(limb darkening)''と呼ばれるもので、光球の内部では高度が高くなるほど温度が下がるために生じる。周縁ではより高い高度の冷たいガスを通して太陽の光が届くため、暗く見えるのだ。これは太陽の温度構造を示す重要な手がかりとなる。また、光球には黒点や白斑と呼ばれる特徴的な構造が現れることがある。黒点は周囲よりも温度が低いため暗く見え、白斑はその逆に温度が高いため明るく見える。これらの構造は太陽の磁場と深く関係している。
\begin{figure}[H]
	\centering
	\includegraphics[width=0.6\linewidth]{Chapter2/Figures/ch2_solar_granule.jpg} 
	\caption{左図はひので衛星可視光望遠鏡による粒状班。観測波長はGバンド(430nm)、画像の縦横幅は4万km。右図は粒状班の模式図(国立天文台/JAXA\cite{hinode_photosphere})。}
	\label{fig:solar_granule}
\end{figure}



図~\ref{fig:solar_granule} に、太陽表面の細かい模様である``粒状斑(granule)''を示す。太陽の光球を拡大すると、一見なめらかに見える表面が細かい斑点のような構造に覆われていることがわかる。これは太陽内部の対流運動によって生じるもので、熱いガスが中心部から上昇し、冷えたガスが周囲に下降することで形成される。粒状斑の典型的なサイズは約1,000kmであり、非常に短い寿命を持つ。平均すると6-10分程度で形が変わるため、まるで太陽表面が常に沸騰しているように見える。

さらに、粒状斑よりもはるかに大きなスケールの対流構造として``超粒状斑(supergranule)''がある。これは1962年にレイトンによって発見されたもので、直径が約3万kmにも達する。超粒状斑は光球全体に広がっており、粒状斑よりもゆっくりとした対流運動を示す。この対流によって太陽表面の磁場が移動し、黒点やフレアの発生にも関与していると考えられている。光球の対流構造は、太陽内部のエネルギー輸送に関わる重要な現象である。これを理解することで、太陽の活動や磁場の変化、さらには宇宙天気への影響を解明する手がかりとなる。

\subsection{彩層}

図~\ref{fig:solar_choromosphere} に、黒点周辺の光球と彩層の様子を示す。太陽の光球の外には、温度が数千Kから1万K程度に達する``彩層(chromosphere)''という大気層が存在する。彩層は、歴史的には皆既日食時にほんの数秒間だけ見られる薄いピンク色の光の層として認識されていた。その鮮やかな色が、この層に``彩層''という名前をつけた理由である。
彩層の温度は、ちょうど水素の電離が始まる境界付近であり、そのため水素のH$\alpha$線(波長6563Å)によってはっきりと観測される。このH$\alpha$線を使った観測では、光球とは異なる太陽の特徴を捉えることができる。

黒点がある領域(活動領域)をH$\alpha$線で観察すると、白色光では見られないような筋状の構造が見て取れる。これらの筋状構造は、太陽内部の磁場の影響を強く受けたプラズマの動きによって生じており、磁力線に沿って分布している。白色光で見る光球は主に連続的なスペクトルを放射するのに対し、H$\alpha$線を使った観測では、彩層特有の輝線放射を捉えることができる。このため、磁場が影響を与えているプラズマの動きを詳しく解析することが可能となる。H$\alpha$線による観測は、太陽の大気におけるエネルギーの移動や磁場構造を理解するために非常に重要である。また、フレアやプロミネンスなど、太陽活動の現象を研究する際にも欠かせない観測方法となっている。



\begin{figure}[H]
	\centering
	\includegraphics[width=0.8\linewidth]{Chapter2/Figures/ch2_solar_chromosphere.jpg} 
	\caption{GバンドとH$\alpha$で見た、黒点周辺の光球・彩層の様子(国立天文台/JAXA\cite{hinode_sunspot})。}
	\label{fig:solar_choromosphere}
\end{figure}

彩層では、光球よりもさらに複雑で多様な構造が見られる。特に黒点周辺では、H$\alpha$線で明るく見える```プラージュ(plage)''と呼ばれる領域や、黒い筋状の``ダーク・フィラメント''と呼ばれる構造が目を引く。これらは、磁場とガスの圧力がバランスを取っていることで形成されている。フレアなどの活動現象も彩層内で頻繁に発生し、これらの動きや構造は太陽内部のダイナミクスに深く関わっている。

\subsection{コロナ}

コロナは太陽の外側に広がる非常に高温の大気層で、彩層のさらに外側に位置している。このコロナは、薄い遷移層を通過して、200万Kという非常に高い温度を持っている。皆既日食時に見られる真珠色に輝くコロナの様子は、古代から人類を魅了してきたが、その明るさは太陽全体の100万分の1ほどで、満月の明るさとほぼ同じくらいである。このため、コロナは地上では通常見ることができず、皆既日食時や特殊な観測装置``コロナグラフ''を使うことでしか観測できなかった。
しかし、1960年代から人工衛星によるX線観測が始まり、そのおかげでコロナについての理解が急速に進んだ。X線を使った観測によって、コロナがループ状の構造を持っていることがわかった。また、コロナの明るさが均一ではなく、黒点が集中している活動領域では特に明るく、逆に北極や南極付近では暗いことも分かった。極域付近の特に暗い部分は``コロナホール(coronal hole)''と呼ばれ、この場所では磁力線が太陽系外に開いており、プラズマが太陽風として外に逃げているため、プラズマ密度が周りよりも低くなっている。このようなコロナホールは極域に限らず、低緯度帯でもしばしば現れる。

コロナの温度がなぜ200万Kもの高温であるのかは、長年の謎であり、これを``コロナ加熱問題''と呼ぶ。この問題は1940年代に初めて発見されてから、半世紀以上経った現在でも解決されていない。しかし、コロナの加熱には磁場が重要な役割を果たしていることは明らかである。コロナ内で明るいループ状の構造が活動領域に集中し、磁場の弱い極域では暗く見えることからも、その関連性が示唆されている。
`

\newpage
\section{太陽活動}

太陽大気では、太陽黒点の近傍での突発的な爆発現象やジェット現象など、あらゆる活動現象が太陽大気で観測されている。

\subsection{黒点と白斑}

図~\ref{fig:sunspot_tube} に、巨大な磁気チューブが浮上して黒点が形成される様子を示す。黒点(sunspot)は、太陽の表面で見られる暗い構造で、白色光で観測するとシミのように見える。黒点の典型的な大きさは数万キロメートル、つまり地球と同じくらいの大きさである。黒点が暗く見えるのは、周囲の光球より温度が少し低いためで、約4,000Kの温度を持っている。しかし、黒点が暗いと言っても、実際には満月よりも数万倍も明るい。
黒点の構造は、中心部分が特に暗い``暗部''と、その周りに放射状に並ぶ明るい部分``半暗部''がある。半暗部は明暗の筋状構造を持ち、灰色に見えることが特徴である

\begin{figure}[H]
	\centering
	\includegraphics[width=0.5\linewidth]{Chapter2/Figures/ch2_sunspot_tube.jpg} 
	\caption{巨大磁気チューブの浮上による黒点の形成(国立天文台/JAXA\cite{hinode_sunspot_model})。}
	\label{fig:sunspot_tube}
\end{figure}


黒点は、太陽内部から浮上してきた巨大な磁場のチューブによって形成される。図~\ref{fig:sunspot_tube}に示されるように、この磁場は光球で断面を作り、強い磁場を持つ黒点を生み出す。大きな黒点ほど、より強い磁場を持つ傾向がある。黒点は通常、2つの黒点が対になって現れる。これらの黒点は、太陽の自転方向に基づき、西側のものを``先行黒点''、東側のものを``後行黒点''と呼ぶ。
黒点にはいくつかのタイプがあり、形や大きさ、磁場の配置によって分類される。図~\ref{fig:sunspot_Mointosh}に示すチューリッヒ分類では、黒点の成長や消滅過程、サイズ、複雑さに基づいて7つのタイプに分類されている。この分類方法は太陽フレアの予測にも役立つ。また、マウントウィルソン分類では、磁場極性の分布の複雑さに応じて、$\alpha$型(単極群)、$\beta$型(双極郡)、$\gamma$型(複雑群)、$\delta$型()密接複雑群)に分類される。特に$\delta$型は磁場が非常に複雑で、こうした黒点を持つ活動領域では大規模なフレアが発生することがある。一方、白斑(facula)は、太陽の縁付近で見られる白くぼんやりとした点々模様である。これらは強い磁場を持つ微細な磁束管の断面で、直径は約100kmほどである。

		
\begin{figure}[H]
	\centering
	\includegraphics[width=0.7\linewidth]{Chapter2/Figures/ch2_sunspot_MoIntosh.jpg} 
	\caption{チューリッヒ分類(国立科学博物館\cite{Zurich})}
	\label{fig:sunspot_Mointosh}
\end{figure}		
		
\begin{table}[htbp]
	\begin{center}
		\caption{チューリッヒ分類とその特徴(国立科学博物館\cite{Zurich})}
		\label{tb:flare_SiLydivSiHe}
		\begin{tabular}{lll}
			\hline \hline
			タイプ & 特徴 \\\hline
			A型 & 半暗部のない単一黒点 \\
			B型 & 半暗部のない黒点群で、双極性がある\\
			C型 & 双極性の黒点群で、一方の主黒点に半暗部がある\\
			D型 & 双極性の黒点群で、両方の主黒点に半暗部がある\\
			E型 & 大きい双極群で、半暗部をもつ2つの主黒点の間に小黒点が散在している\\
			F型 & 非常に大きな双極または複雑な黒点群\\
			G型 & 半暗部をもつ大きな双極群で、その間に小黒点が散在しない\\
			H型 & 半暗部をもつ単極性の黒点で、直径が2.5度以上ある\\
			J型 & 半暗部をもつ単極性の黒点で、直径が2.5度未満
			\\\hline
		\end{tabular}
	\end{center}
\end{table}	 

\begin{comment}
\subsection{プロミネンスとダーク$\cdot$フィラメント}
プロミネンス(prominence)は、古くは、皆既日食の際に太陽の縁から立ち上る赤い炎のような構造として知られてきた。プロミネンスの正体は、数千から 1 万 K 程度の冷たいプラズマであり、水素原子からの H$\alpha$線スペクトルを放射するため、赤っぽく光っている。ただしプロミネンスが放つ光はそれほど強くないため、太陽の縁上では明るく見えるが、太陽面上にくると黒い筋模様(ダーク$\cdot$フィラメント)として見える。プロミネンスのプラズマは、磁
場の力によってコロナ中に浮かんでいる。活動領域から離れた場所に現れるものは静穏型プロミネンスと呼ばれ、数ヵ月にわたって安定に存在するものもある。一方で、活動領域内に現れる活動領域型プロミネンスは、一般に短寿命で、フレアに伴って噴出するものも見られる。 

\begin{figure}[H]
	\centering
	\includegraphics[width=0.6\linewidth]{Chapter2/Figures/ch2_prominence.jpg} 
	\caption{左図は、NASAの太陽観測衛星SDOが極端紫外光でとらえた太陽全面画像。右図は、太陽観測衛星ひのでが可視光で撮影した太陽プロミネンス。プロミネンスの細長い筋状の構造が現れている。同じ縮尺の地球と比較すると、プロミネンスの巨大さが明らかである。(NASA/JAXA/NAOJ\cite{SDO_prominence}}
	\label{fig:prominence}
\end{figure}	
\end{comment}

\newpage
\subsection{太陽フレア}
太陽フレアは、太陽の大気で蓄積された磁気エネルギーが急激に放出される爆発的な現象である。この現象では、短時間のうちに非常に多くのエネルギーが放出され、地球に届く電磁波の強度が急激に増加する。
図~\ref{fig:flare_lcurve_vari_wave}に、さまざまな波長での放射強度の光度曲線を示す。この光度曲線は、太陽フレアや他の天体のフレア活動のダイナミクスを理解する上で重要である。特に、異なる波長帯域における放射強度の変化は、フレアが放出するエネルギーの異なる部分を反映しており、フレアの発生メカニズムやその後の進化を解析するための手がかりを提供する。

\begin{figure}[H]
	\centering
	\includegraphics[width=0.6\linewidth]{Chapter2/Figures/ch2_flare_lcurve_vari_wave.jpg} 
	\caption{典型的なフレアにおける、さまざまな波長での放射強度の光度曲線(Kane et al.1974 \cite{Kane})}
	\label{fig:flare_lcurve_vari_wave}
\end{figure}

\newpage

図~\ref{fig:flare_ribbon}に可視光磁場望遠鏡がとらえたXクラスフレアを示す。太陽フレアは主に黒点周辺で発生する。黒点は強い磁場を持つ領域であり、ここで磁気エネルギーが蓄積される。何らかのきっかけでそのエネルギーが解放されると、フレアが発生する。このエネルギー解放のメカニズムとして、磁気リコネクションという現象がよく説明される。
フレアの規模はさまざまで、小さなものから非常に大きなものまである。特に大規模なフレアでは、10万km四方以上の広い範囲でエネルギーが放出され、その影響は太陽系全体に及ぶ可能性もある。そのため、フレアの発生をリアルタイムで監視し、その影響を予測することは宇宙天気予報の面で非常に重要である。

フレアの観測では、特定の波長で撮影を行うことでその構造を詳細に捉えることができる。図~\ref{fig:flare_ribbon}に示されるように、可視光のH$\alpha$線で観測すると、フレアの発生した場所に沿って細長い明るい構造が現れる。この構造はフレアリボンと呼ばれ、しばしば二つのリボンが並んで見えることがある。このようなフレアはツーリボン・フレアと呼ばれ、典型的な磁気リコネクションの特徴を持っている。
フレアリボンは、太陽の表面において異なる磁場の極性を持つ二つの領域に現れる。具体的には、一方のリボンはN極の磁場を持つ領域に、もう一方はS極の磁場を持つ領域に位置する。このことから、フレアが起こる原因が太陽コロナ中に蓄積された磁気エネルギーであり、その解放過程で磁力線が再結合することがわかる。
フレアが進行するにつれて、フレアリボンは時間とともに互いに離れていく。この広がりの速さは通常10-100 km/sであり、磁気リコネクションで解放されるエネルギーの規模によって異なると考えられている。この広がりの速さは、フレアのエネルギー放出の進行具合を示す重要な指標となっている。

\begin{figure}[H]
	\centering
	\includegraphics[width=0.6\linewidth]{Chapter2/Figures/ch2_flare_ribbon.jpg} 
	\caption{可視光磁場望遠鏡がとらえたXクラスフレア。上はカルシウムH線、下は可視光連続光。JAXA\cite{SOT}}
	\label{fig:flare_ribbon}
\end{figure}


図~\ref{fig:flare_solar-a}に、太陽の縁付近で発生したフレアのX線画像が示されている。X線で観測すると、フレアに伴って明るく輝くループ状の構造が見える。このループはフレアループと呼ばれ、特にその上部が鋭く尖った形状のものはカスプ構造として知られている。このようなカスプ型フレアでは、エネルギーの解放が太陽コロナ中で発生していることを示唆している。フレアの進行に伴い、このカスプ構造は成長し、高温で高エネルギーの領域に発展していく。
\begin{figure}[H]
	\centering
	\includegraphics[width=0.6\linewidth]{Chapter2/Figures/ch2_falre_solar-a.jpg} 
	\caption{1992年2月2日のフレアの``ようこう''軟X線望遠鏡による観測。太陽の東の縁付近で起きたフレアで、上が太陽面の東、右が北。太陽面上では1秒角が700kmにあたる。ISAS/JAXA\cite{youko_flare_loop}}
	\label{fig:flare_solar-a}
\end{figure}


図~\ref{fig:flare_ribbon}に、太陽フレアのエネルギー解放過程を説明する理論の一つである磁気リコネクションモデルを示す。このモデルでは、コロナ中に逆向きの磁力線が圧縮され、臨界点でそれらが再接続(リコネクション)する。これにより、蓄積されていた磁場エネルギーが急激に解放され、プラズマの運動エネルギーや熱エネルギーに変換される。カスプ型フレアループは、この磁気リコネクションが示す磁場構造を反映している。さらに、フレアループは単独で存在するわけではなく、複数のループがアーケード状に並ぶことが多い。H$\alpha$線で観測されたフレアリボンは、これらのフレアループの足元に対応していると考えられている。フレアが進行するにつれて、フレアループとフレアリボンの間隔が広がることは、磁気リコネクションが継続して発生し、新しい磁場構造が形成されていることを示している。これらの観測結果から、磁気リコネクションが太陽フレアの主な駆動メカニズムであることが確認されている。

\begin{figure}[H]
	\centering
	\includegraphics[width=0.5\linewidth]{Chapter2/Figures/ch2_plasmoid_ejection_model.jpg} 
	\caption{フレアの磁気リコネクションモデル。実線は磁力線を表す。Shibata et al.1995\cite{Shibata}}
	\label{fig:flare_ribbon}
\end{figure}

\begin{comment}
\newpage
\subsection{コロナ質量放出(CME)}
人工衛星による観測技術の発達に伴い、1970 年代からは宇宙空間での人工日食によるコロナの観測が可能となった。それらにより、突然大量のコロナガスが太陽から惑星間空間へ放出される現象が発見された。これをコロナ質量放出(coronal mass ejection; CME)と呼ぶ(図21)。放出される質量は 10-100億トン(1015-1016g)、速度は 10-3000km/s (平均速度は 500km s-1)、大きさは太陽半径の数倍から 10 倍にも及ぶ。大きなフレアに伴って発生することが多いが、中にはフレアが起きていなくても発生することがある。そのような場合は、フィラメント(プロミネンス)の噴出が起きていることが多い(図22)。逆に、小規模なフレアの場合、CME を伴わないことも多い。CME や太陽風は太陽から大量のガスや磁力線を放出しており、惑星間空間に多大な影響を及ぼしている。

 \begin{figure}[H]
	\centering
	\includegraphics[width=0.6\linewidth]{Chapter2/Figures/ch2_CME.jpg} 
	\caption{2000年2月27日のCMEを SOHO LASCO C2 と C3 が撮影した画像。(SOHO/ESA\&NASA)}
	\label{fig:CME}
\end{figure}

プロミネンスやフィラメントの噴出は、フレア(磁気リコネクション)にともなって解放される磁気エネルギーにより上方に加速されたと考えると説明が付く。噴出速度は 10-500km/s とさまざまある。また、加速の継続時間も数分から数時間と幅広く分布する。フィラメント噴出では、H$\alpha$線で見える構造だけでなく、周辺の 100 万 K コロナプラズマも一緒に噴出する。これが CME であると考えられる。 

 \begin{figure}[H]
	\centering
	\includegraphics[width=0.5\linewidth]{Chapter2/Figures/ch2_prominence_eruption.jpg} 
	\caption{2022年2月15日に発生した太陽プロミネンスの噴出を、ESAのソーラー・オービターに搭載されたEUIのFSIが撮影した画像。(Solar Orbiter/EUI and SOHO/LASCO teams, ESA \& NASA\cite{ESA_EUI_FSI})}
	\label{fig:prominence_eruption}
\end{figure}
\end{comment}
	
\newpage
\subsection{太陽活動周期とダイナモ}
太陽の黒点や太陽フレアは一定の周期で発生し、その数は時間とともに増減する。この変動はおおよそ11年周期で繰り返されることが知られており、これを「太陽活動周期」と呼ぶ。黒点の数は太陽の活動を示す重要な指標であり、特に太陽の磁場の動きと深く関わっている。太陽活動の指標として使われる黒点相対数は、黒点群の数($g$)と、その群に含まれる黒点の総数($f$)を使って計算される。具体的には、黒点相対数 $R$ は$ R = k(10g + f) $ と定義される。ここで、$k$ は観測機器や観測者ごとの違いを補正するための係数である。これにより、異なる時期や場所での観測結果を比較できるようになる。

黒点相対数の変動は太陽の磁場構造の変化を反映している。太陽活動が活発な時期には黒点数が増え、それに伴ってフレアが多く発生する。一方、活動が低下すると黒点数も減少し、フレアの発生も少なくなる。太陽活動周期は単純な規則性に従うわけではなく、その振幅や持続時間は時期ごとに変動することが分かっている。
黒点相対数の変化を詳しく調べることで、太陽の磁場活動がどのように進んでいるかを理解できるだけでなく、その影響が地球の環境にどのように現れるかを予測することも可能になる。最近では、過去の黒点データを使って太陽活動の長期的な変動を解析し、太陽の磁場の変化と気候の変動との関係についても議論が進められている。

\begin{figure}[H]
	\centering
	\includegraphics[width=0.8\linewidth]{Chapter2/Figures/ch2_solar_spot_trend.jpg} 
	\caption{過去400年にわたる太陽黒点相対数の変化。JAXA\cite{solar_system}}
	\label{fig:solar_spot_trend}
\end{figure}

図~\ref{fig:solar_spot_trend}に示す通り、太陽黒点の相対数は1700年ごろから約11年周期で増減を繰り返している。この周期的な変動は``太陽周期''と呼ばれ、黒点の数が最大になる時期を``極大期''、最小になる時期を``極小期''と区別する。太陽活動はこの周期に従って変化し、極大期には黒点の数が増え、太陽フレアなどの活発な現象が観測される傾向がある。
黒点の出現位置にも特徴があり、太陽周期の初めでは高緯度の領域に黒点が現れやすい。しかし、時間が経過するにつれて、黒点は徐々に低緯度の方へと移動していく。この現象を時系列で観察したものが、図~\ref{fig:sunspot_trend}に示されるプロットである。この図では、太陽面上での黒点の出現緯度を縦軸、時間を横軸にとり、11年周期ごとに蝶の羽のような形状を描いている。この特有のパターンは``蝶型図''と呼ばれ、黒点の動きや形成過程を視覚的に理解するために重要な役割を果たしている。
黒点相対数の周期的な変動やその移動パターンは、太陽活動の基本的な特徴を示しており、太陽内部で起こるダイナモ機構と強く関連していると考えられている

\begin{figure}[H]
	\centering
	\includegraphics[width=0.8\linewidth]{Chapter2/Figures/ch2_sunspot_trend.jpg} 
	\caption{黒点蝶形図。NASA\cite{sunspot_cycle}}
	\label{fig:sunspot_trend}
\end{figure}

太陽黒点の活動周期は約11年だが、黒点の磁場の極性に着目すると、この周期は実際には22年であることがわかる。黒点は通常、西側(先行黒点)と東側(後行黒点)のペアとして現れ、その磁場の極性は南北半球で反転している。各活動周期内では、同じ半球で同じ極性を持つ黒点ペアが現れるが、次の周期が始まると、黒点ペアの磁場極性が反転する。この現象は``ヘールの法則''として知られており、黒点の磁場反転が周期的に繰り返されることを示している。

黒点がどのようにして形成されるのかという答えの一つは、太陽の差動回転にある。太陽の赤道付近は極域よりも回転が速く、この差動回転の影響で、本来南北方向にまっすぐだった磁力線が徐々に引き伸ばされ、最終的に東西方向へと巻き込まれていく。この過程が繰り返されることにより、東西方向にリング状の磁束管ができる。この磁束管が太陽表面に浮かび上がり、黒点が形成されると考えられている。これにより、先行黒点と後行黒点が南北半球で磁場が逆転する仕組みも理解できる。

また、南北方向の磁場は``ポロイダル磁場''と呼ばれ、東西方向の磁場は``トロイダル磁場''と呼ばれる。ポロイダル磁場がトロイダル磁場に変わる過程は``$\omega$効果''として知られ、この効果は太陽の差動回転が主な原因となる。しかし、太陽の磁場は単に引き延ばされるだけでなく、11年ごとに南北方向の磁場が反転する。これを説明するには、追加のメカニズムが必要となる。その役割を果たすのが``$\alpha$効果''であり、太陽内部のコリオリ力や乱流によって引き起こされると考えられている。しかし、$\alpha$効果がどのように、そしてどの場所で発生するのかについては未解明の部分が多く、さらなる研究が求められている。
`
\begin{figure}[H]
	\centering
	\includegraphics[width=0.8\linewidth]{Chapter2/Figures/ch2_solar_diff_rotation.jpg} 
	\caption{太陽の差動回転の模式図。南北に走る磁力線が、差動回転により東西方向に引き延ばされる。JAXA\cite{solar_system}}
	\label{fig:soalr_diff_rotation}
\end{figure}


\newpage
\section{太陽フレアにおけるX線放射}

\begin{figure}[H]
	\centering
	\includegraphics[width=0.8\linewidth]{Chapter2/Figures/ch2_flare_reshii_spec.jpg} 
	\caption{RHESSI 衛星によって観測された複合的なフレアスペクトル。Lin et al.2002\cite{Lin}}
	\label{fig:ch2_flare_reshii_spec}
\end{figure}

太陽フレアの際に放射されるX線は、主に2つのメカニズムによって発生する。一つは、高温プラズマから放射される``熱的制動放射''、もう一つは、高エネルギーの電子ビームがプラズマ中のイオンとクーロン衝突を起こして放射される``非熱的制動放射''である。図~\ref{fig:ch2_flare_reshii_spec}に、RHESSI 衛星によって観測された複合的なフレアスペクトルと``熱的制動放射''、``非熱的制動放射''によって放射される領域を示す。特に、非熱的制動放射については、電子ビームがどのような環境でエネルギーを失うかによって、2つの異なるモデルが考えられている。それが``厚い標的モデル(thick-target model)''と``薄い標的モデル(thin-target model)''である。

厚い標的モデルでは、高エネルギーの電子が高密度のプラズマ中を移動しながら繰り返しクーロン衝突を起こし、急速にエネルギーを失うと考える。このモデルは、電子ビームが磁気ループのフットポイントに向かって進行するときに適用される。フットポイントでは磁場が強く、プラズマ密度が高いため、電子はすぐにエネルギーを失い、そこから強いX線が放射される。

薄い標的モデルでは、電子が比較的低密度のプラズマ中を移動し、制動放射によって徐々にエネルギーを失うと考える。このモデルは、電子ビームが磁気ループ内で反射されたり、ループにトラップされたりする場合に適用される。ループ内の密度が低いため、電子はすぐにはエネルギーを失わず、何度も反射を繰り返しながら少しずつ制動放射を行う。

太陽フレアにおけるプラズマの密度は場所によって大きく異なる。例えば、``彩層(chromosphere)''と``コロナ(corona)''では、密度の比率が約 $3 \times (10^2 - 10^3)$ 程度とされる。そのため、多くの電子はフットポイントのような高密度領域でエネルギーを急速に失い、厚い標的モデルに従って放射を行うことが多い。


\newpage
\subsection{非熱的制動放射}
加速された電子が放射するX線スペクトルは、電子がどのようにターゲット物質と相互作用するかによって異なる。特に、電子が厚いターゲット内で減速しながら放射を行う``thick-target モデル''と、薄いターゲットを一度だけ通過して放射する``thin-target モデル''では、得られるX線のスペクトルが変化する。

加速電子の入射時のエネルギースペクトルを
\begin{equation}
	F(E_0) = AE^{-\delta} \quad \text{(electrons/s/keV/cm$^2$)} \quad 
\end{equation}
と仮定する。

このとき、thick-target モデルにおける観測X線スペクトルは、べき関数の形をとり
\begin{equation}
	I_{\text{thick}}(\varepsilon) = A_{\text{thick}} \varepsilon^{-\gamma} \quad \text{(photons/s/keV/cm$^2$)} \quad 
\end{equation}
と表される。

スペクトルの強度を決める係数$a_{thick}$は
\begin{equation}
	%a_{thin} = \frac{A_{BH} Z^2 S \Delta N}{4 \pi R^2 K} B(\delta, 1) \quad 
	a_{thick} = \frac{SA\kappa_{BH}Z^{2}}{4\pi R^{2}K} \cdot \frac{B(\delta-2,\frac{1}{2})}{(\delta-2)(\delta-1)}
\end{equation}

また、スペクトルのベキ指数$\gamma$は
\begin{equation}
	\gamma = \delta - 1 \quad 
\end{equation}

で与えられる。ここで、$\kappa_{BH}$は、古典的な制動放射断面積の係数$\kappa_{BH} = \frac{8}{3} r_0^2 m_e c^2 = 7.9 \times 10^{-25} \text{cm}^2 \text{keV}$ である($\alpha = \frac{e^2}{\hbar c}$: 微細構造定数、$r_0 = \frac{e^2}{m_e c^2}$: 古典電子半径))、\mbox{$Z$: 標的となるイオンの原子量}、$K = 2 \pi e^4 \ln \Lambda$ ($\Lambda$:クーロン対数)、$R = 1 \text{AU}$ で、$B(p,q)$ はベータ関数である。
この結果は、電子が厚いターゲット内で複数回のクーロン衝突を繰り返しながら制動放射を行うため、より高エネルギーのX線が多く放射されることに起因する。従って、thick-target モデルでは、電子のスペクトルよりもハードなX線スペクトルが得られる。


thin-target モデルでは、電子がターゲットを一度だけ通過し、その過程で制動放射によってX線を放つ。入射電子のエネルギースペクトルが
\begin{equation}
	F(E_0) = AE^{-\delta} \quad (\text{photons/s/keV/cm}^2) \quad 
\end{equation}

\begin{equation}
	I_{thin}(\varepsilon) = a_{thin} \varepsilon^{-\gamma} \quad (\text{photons/s/keV/cm}^2) \quad 
\end{equation}
となる。係数$a_{thick}$は
\begin{equation}
	%a_{thin} = \frac{A_{BH} Z^2 S \Delta N}{4 \pi R^2 K} B(\delta, 1) \quad 
	a_{thick} = \frac{A\kappa_{BH}Z^{2}S\Delta N}{4\pi R^{2}K} \cdot \frac{B(\delta,\frac{1}{2})}{\delta}
\end{equation}
で与えられる。
また、スペクトルのべき指数$\gamma$は
\begin{equation}
	\gamma = \delta + 1 \quad 
\end{equation}
となる。つまり、thin-target モデルでは、電子スペクトルの指数よりもX線スペクトルの指数が 1 ソフトになる。
一方、thick-target モデルでは、電子がターゲット内部で複数回のクーロン衝突を繰り返しながら制動放射を行う。同じ電子スペクトルから放射されるX線を比べると、thick-target モデルの方がthin-target モデルよりもX線スペクトルがハードになる。実際の観測データのX線スペクトルがべき型で表される場合、そのべき指数を測定することで、どちらのモデルが適しているかを判断し、入射電子のスペクトルを推定することができる。


\subsection{熱的制動放射}

高温プラズマ中では、電子が熱運動しており、その運動する電子がイオンと相互作用すると制動放射によってX線が放射される。これを``熱的制動放射(thermal bremsstrahlung)''という。

プラズマ中の電子のエネルギー分布は、Maxwell 分布に従う。温度$T$の電子の分布関数は
\begin{equation}
	N(E) = \frac{2n_e E^{\frac{1}{2}}}{\sqrt{\pi (k_B T)^{\frac{3}{2}}}} \exp \left( -\frac{\varepsilon}{k_B T} \right) \quad (\text{electrons/cm}^3/\text{erg}) \quad
\end{equation}
となる。この分布をもとに、プラズマから放射されるX線スペクトルは
\begin{equation}	
	\begin{aligned}
		I_{thermal}(\varepsilon) = a_{thermal} \frac{1}{\varepsilon} \exp \left( -\frac{\varepsilon}{k_B T} \right) \quad (\text{photons/s/erg}) \\
		a_{thermal} = \left( \frac{8}{\pi m_e k_B T} \right)^{\frac{1}{2}} \kappa_{BH} Z^2 VEM
	\end{aligned}
\end{equation}

となる。ここで $VEM = n_e^2V$ はエミッションメジャーであり、電子密度の二乗とプラズマの体積$V$の積で表される、$n_e$は電子密度、$\kappa_B$はボルツマン定数である。この式からわかるように、熱的制動放射によるX線スペクトルは、エネルギーが高くなるにつれて指数関数的に減衰する。このため、熱的制動放射によるX線スペクトルは高エネルギーほどソフトな形状を持つ。特に、エネルギー分布の指数が温度$T$に依存しており、温度が高いほど高エネルギー側のスペクトルが伸びる。これにより、X線スペクトルの形状を分析することで、プラズマの温度を推定することが可能となる。

