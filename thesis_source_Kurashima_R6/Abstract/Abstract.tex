\section*{Abstract}

本研究では、X線分光撮像衛星 XRISM に搭載した 軟X線撮像検出器 SXI を用いて、\mbox{2023年10月19日}から\mbox{2024年12月31日}の期間における、太陽フレアの地球大気反射X線 の観測を行った。太陽活動の極大期により、\mbox{2024年}に大規模な太陽フレアが頻繁に発生し、XRISM /SXI でも多数のフレアを観測した。観測したフレアから、フレアピーク付近を観測した9つのフレアを解析に使用した。まず、9つのフレアを足し合わせた積分スペクトルを作成した。フレアの積分スペクトルから、Si, S, Ar, Ca, Ti, Cr, Fe, Ni の輝線を検出した。Katsuda et al. 2020\cite{Katsuda}と比較して、新たに Ti, Cr, Ni の輝線が得られた。次に、フレアクラス別に Si, S, Ar, Ca, Fe の絶対元素組成比の測定を行った。測定の結果、すべてのフレアで、Katsuda et al. 2020\cite{Katsuda}と同様に、活動的な恒星コロナに見られる i-FIP 効果に類似する絶対元素組成比パターンが得られた。また、本研究で新たに、Fe の絶対元素組成比の測定を行った結果、ほぼ全てのフレアにおける絶対元素組成比が 0.76-1.37 であった。これは、FIP の近い Si の絶対元素組成比と同様の結果であり、恒星コロナの FIP 効果と i-FIP 効果を説明する  Laming model (Laming et al, 2021\cite{Laming_2021})と整合している。最後に、フレアピーク付近における絶対元素組成比の時間変動を検証した。Si , S, Ca の絶対元素組成比は、フレアの立ち上がりから、フレアピークにかけて増大し、フレアピーク以降は減少している傾向が見られた。特に Ca はフレアピーク付近で顕著に変動することが明らかとなった。